

\documentclass[DM,authoryear,toc]{lsstdoc}
% lsstdoc documentation: https://lsst-texmf.lsst.io/lsstdoc.html

% Package imports go here.

% Local commands go here.

% To add a short-form title:
% \title[Short title]{Title}
\title{S17B HSC PDR1 Reprocessing Report}

% Optional subtitle
% \setDocSubtitle{A subtitle}

\author{%
Hsin-Fang Chiang, Greg Daues, and the NCSA team
}

\setDocRef{TEST-31}

\date{\today}

% Optional: name of the document's curator
% \setDocCurator{The Curator of this Document}

\setDocAbstract{%
This document captures information about the large scale HSC reprocessing we performed in Cycle S17B.
}

% Change history defined here.
% Order: oldest first.
% Fields: VERSION, DATE, DESCRIPTION, OWNER NAME.
% See LPM-51 for version number policy.
\setDocChangeRecord{%
  \addtohist{1}{YYY-MM-DD}{Unreleased.}{Hsin-Fang Chiang}
}

\begin{document}

% Create the title page.
% Table of contents is added automatically with the "toc" class option.
\maketitle

% ADD CONTENT HERE
\section{Dataset Information}
The input dataset is the HSC Strategic Survey Program (SSP) Public Data Release 1 (PDR1) \citep{2017arXiv170208449A}.
The PDR1 dataset has been transferred to the LSST GPFS storage /datasets by \jira{DM-9683} and the butler repo is available at /datasets/hsc/repo.

It includes 5654 visits in 7 bands: HSC-G, HSC-R, HSC-I, HSC-Y, HSC-Z, NB0816, NB0921. Their visit IDs are visitId-SSPPDR1.txt.  The official release site is at https://hsc-release.mtk.nao.ac.jp/
The survey has three layers and includes 8 fields.
\begin{enumerate}
\item
UDEEP: SSP{\_}UDEEP{\_}SXDS, SSP{\_}UDEEP{\_}COSMOS
\item
DEEP: SSP{\_}DEEP{\_}ELAIS{\_}N1, SSP{\_}DEEP{\_}DEEP2{\_}3, SSP{\_}DEEP{\_}XMM(S){\_}LSS, SSP{\_}DEEP{\_}COSMOS
\item
WIDE: SSP{\_}WIDE, SSP{\_}AEGIS
\end{enumerate}

The number of visits in each field and band is summarized in the following table.

%Summary of the input dataset
\begin{tabular}{lllllllll}
& & HSC-G & HSC-R & HSC-I & HSC-Z & HSC-Y & NB0921 & NB0816  \\
Layer & Field Name ("OBJECT") & \multicolumn{7}{c}{Number of visits} \\
DEEP & SSP{\_}DEEP{\_}ELAIS{\_}N1 & 32&24&28&51&24&20&0 \\
DEEP & SSP{\_}DEEP{\_}ELAIS{\_}N1 &32&24&28&51&24&20&0 \\
DEEP & SSP{\_}DEEP{\_}DEEP2{\_}3 &32&31&32&44&32&23&17 \\
DEEP & SSP{\_}DEEP{\_}XMM{\_}LSS &25&27&18&21&25&0&0 \\
DEEP & SSP{\_}DEEP{\_}COSMOS &20&20&40&48&16&18&0 \\
UDEEP&SSP{\_}UDEEP{\_}SXDS&18&18&31&43&46&21&19 \\
UDEEP & SSP{\_}UDEEP{\_}COSMOS&19&19&35&33&55&29&0 \\
WIDE&SSP{\_}AEGIS&8&5&7&7&7&0&0 \\
WIDE & SSP{\_}WIDE&913&818&916&991&928&0&0 \\
\end{tabular}


The tract IDs for each field, obtained
from https://hsc-release.mtk.nao.ac.jp/doc/index.php/database/
is summarized in the following table.

%Summary of the input dataset
\begin{tabular}{lll}
Layer & Field Name ("OBJECT") & Tract IDs \\
DEEP & SSP{\_}DEEP{\_}ELAIS{\_}N1 & 16984, 16985, 17129, 17130, 17131, 17270, 17271, 17272, 17406, 17407 \\
DEEP & SSP{\_}DEEP{\_}DEEP2{\_}3 & 9220, 9221, 9462, 9463, 9464, 9465, 9706, 9707, 9708 \\
DEEP & SSP{\_}DEEP{\_}XMM{\_}LSS & 8282, 8283, 8284, 8523, 8524, 8525, 8765, 8766, 8767 \\
DEEP & SSP{\_}DEEP{\_}COSMOS & 9569, 9570, 9571, 9572\footnote{tract 9572 is listed on HSC PDR1 website for DEEP{\_}COSMOS but no data actually overlap it; PDR1 does not have it either.}, 9812, 9813, 9814, 10054, 10055, 10056 \\
UDEEP&SSP{\_}UDEEP{\_}SXDS& 8523, 8524, 8765, 8766 \\
UDEEP & SSP{\_}UDEEP{\_}COSMOS& 9570, 9571, 9812, 9813, 9814, 10054, 10055 \\
WIDE&SSP{\_}AEGIS& 16821,16822, 16972, 16973 \\
\multirow{3}{*}{WIDE} & \multirow{3}{*}{SSP{\_}WIDE}
& XMM: 8279-8285, 8520-8526, 8762-8768 \\
&&GAMA09H: 9314-9318, 9557-9562, 9800-9805 \\
&&WIDE12H: 9346-9349, 9589-9592 \\
&&GAMA15H: 9370-9375, 9613-9618 \\
&&HECTOMAP: 15830-15833, 16008-16011 \\
&&VVDS: 9450-9456, 9693-9699, 9935-9941 \\
\end{tabular}


Plots of tracts and patches can be found on  https://hsc-release.mtk.nao.ac.jp/doc/index.php/data/. In S17B, more tracts than listed were processed.

\section{Hardware}
The processing was done using the Verification Cluster.
The Verification Cluster consists of 48 Dell C6320 nodes with 24 physical cores (2 sockets, 12 cores per processor) and 128 GB RAM. As such, the system provides a total of 1152 physical cores.
lsst-dev01 is a system with 24 physical cores, 256 GB RAM, running the latest CentOS 7.x that serves as the front end of the Verification Cluster.

The Verification Cluster runs the Simple Linux Utility for Resource Management (SLURM) cluster management and job scheduling system. lsst-dev01 runs the SLURM controller and serves as the login or head node , enabling LSST DM users to submit SLURM jobs to the Verification Cluster.

lsst-dev01 and the Verification Cluster utilize the General Parallel File System (GPFS) to provided shared-disk across all of the nodes. The GPFS will have spaces for archived datasets and scratch space to support computation/analysis.



\section{Software}


% Include all the relevant bib files.
% https://lsst-texmf.lsst.io/lsstdoc.html#bibliographies
\bibliography{lsst,lsst-dm,refs_ads,refs,books,local}

\end{document}
